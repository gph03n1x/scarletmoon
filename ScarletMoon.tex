%% Generated by Sphinx.
\def\sphinxdocclass{report}
\documentclass[letterpaper,10pt,english]{sphinxmanual}
\ifdefined\pdfpxdimen
   \let\sphinxpxdimen\pdfpxdimen\else\newdimen\sphinxpxdimen
\fi \sphinxpxdimen=49336sp\relax

\usepackage[margin=1in,marginparwidth=0.5in]{geometry}
\usepackage[utf8]{inputenc}
\ifdefined\DeclareUnicodeCharacter
  \DeclareUnicodeCharacter{00A0}{\nobreakspace}
\fi
\usepackage{cmap}
\usepackage[T1]{fontenc}
\usepackage{amsmath,amssymb,amstext}
\usepackage{babel}
\usepackage{times}
\usepackage[Bjarne]{fncychap}
\usepackage{longtable}
\usepackage{sphinx}

\usepackage{multirow}
\usepackage{eqparbox}

% Include hyperref last.
\usepackage{hyperref}
% Fix anchor placement for figures with captions.
\usepackage{hypcap}% it must be loaded after hyperref.
% Set up styles of URL: it should be placed after hyperref.
\urlstyle{same}

\addto\captionsenglish{\renewcommand{\figurename}{Fig.\@ }}
\addto\captionsenglish{\renewcommand{\tablename}{Table }}
\addto\captionsenglish{\renewcommand{\literalblockname}{Listing }}

\addto\extrasenglish{\def\pageautorefname{page}}

\setcounter{tocdepth}{3}
\setcounter{secnumdepth}{3}


\title{Scarlet Moon Documentation}
\date{Jun 16, 2017}
\release{0.1.0}
\author{Giannis Papaioannou}
\newcommand{\sphinxlogo}{}
\renewcommand{\releasename}{Release}
\makeindex

\begin{document}

\maketitle
\sphinxtableofcontents
\phantomsection\label{\detokenize{index::doc}}



\chapter{btree module}
\label{\detokenize{btree:module-btree}}\label{\detokenize{btree:btree-module}}\label{\detokenize{btree::doc}}\label{\detokenize{btree:welcome-to-scarlet-moon-s-documentation}}\index{btree (module)}\index{KeyNode (class in btree)}

\begin{fulllineitems}
\phantomsection\label{\detokenize{btree:btree.KeyNode}}\pysiglinewithargsret{\sphinxstrong{class }\sphinxcode{btree.}\sphinxbfcode{KeyNode}}{\emph{key}, \emph{value}, \emph{term\_freq}}{}
Bases: \sphinxcode{object}
\index{get\_value() (btree.KeyNode method)}

\begin{fulllineitems}
\phantomsection\label{\detokenize{btree:btree.KeyNode.get_value}}\pysiglinewithargsret{\sphinxbfcode{get\_value}}{}{}
\end{fulllineitems}

\index{idf() (btree.KeyNode method)}

\begin{fulllineitems}
\phantomsection\label{\detokenize{btree:btree.KeyNode.idf}}\pysiglinewithargsret{\sphinxbfcode{idf}}{\emph{number\_of\_tokens}}{}
\end{fulllineitems}

\index{update() (btree.KeyNode method)}

\begin{fulllineitems}
\phantomsection\label{\detokenize{btree:btree.KeyNode.update}}\pysiglinewithargsret{\sphinxbfcode{update}}{\emph{value}, \emph{frequency}}{}
\end{fulllineitems}


\end{fulllineitems}

\index{KeyTree (class in btree)}

\begin{fulllineitems}
\phantomsection\label{\detokenize{btree:btree.KeyTree}}\pysiglinewithargsret{\sphinxstrong{class }\sphinxcode{btree.}\sphinxbfcode{KeyTree}}{\emph{parent=None}}{}
Bases: \sphinxcode{object}
\index{add() (btree.KeyTree method)}

\begin{fulllineitems}
\phantomsection\label{\detokenize{btree:btree.KeyTree.add}}\pysiglinewithargsret{\sphinxbfcode{add}}{\emph{key}, \emph{value}, \emph{frequency}}{}
\end{fulllineitems}

\index{delete\_tr() (btree.KeyTree method)}

\begin{fulllineitems}
\phantomsection\label{\detokenize{btree:btree.KeyTree.delete_tr}}\pysiglinewithargsret{\sphinxbfcode{delete\_tr}}{}{}
\end{fulllineitems}

\index{find() (btree.KeyTree method)}

\begin{fulllineitems}
\phantomsection\label{\detokenize{btree:btree.KeyTree.find}}\pysiglinewithargsret{\sphinxbfcode{find}}{\emph{key}}{}
\end{fulllineitems}

\index{get\_doc\_count() (btree.KeyTree method)}

\begin{fulllineitems}
\phantomsection\label{\detokenize{btree:btree.KeyTree.get_doc_count}}\pysiglinewithargsret{\sphinxbfcode{get\_doc\_count}}{}{}
\end{fulllineitems}

\index{traverse() (btree.KeyTree method)}

\begin{fulllineitems}
\phantomsection\label{\detokenize{btree:btree.KeyTree.traverse}}\pysiglinewithargsret{\sphinxbfcode{traverse}}{}{}
\end{fulllineitems}

\index{visualize\_tree() (btree.KeyTree method)}

\begin{fulllineitems}
\phantomsection\label{\detokenize{btree:btree.KeyTree.visualize_tree}}\pysiglinewithargsret{\sphinxbfcode{visualize\_tree}}{\emph{key}}{}
\end{fulllineitems}


\end{fulllineitems}



\chapter{getch module}
\label{\detokenize{getch:getch-module}}\label{\detokenize{getch::doc}}\label{\detokenize{getch:module-getch}}\index{getch (module)}
Using recipe found at code.activestate.com:

\url{http://code.activestate.com/recipes/134892/}
\index{Getch (class in getch)}

\begin{fulllineitems}
\phantomsection\label{\detokenize{getch:getch.Getch}}\pysigline{\sphinxstrong{class }\sphinxcode{getch.}\sphinxbfcode{Getch}}
Bases: \sphinxcode{object}

Gets a single character from standard input.  Does not echo to the screen.

\end{fulllineitems}

\index{GetchUnix (class in getch)}

\begin{fulllineitems}
\phantomsection\label{\detokenize{getch:getch.GetchUnix}}\pysigline{\sphinxstrong{class }\sphinxcode{getch.}\sphinxbfcode{GetchUnix}}
Bases: \sphinxcode{object}

Unix Implementation

\end{fulllineitems}

\index{GetchWindows (class in getch)}

\begin{fulllineitems}
\phantomsection\label{\detokenize{getch:getch.GetchWindows}}\pysigline{\sphinxstrong{class }\sphinxcode{getch.}\sphinxbfcode{GetchWindows}}
Bases: \sphinxcode{object}

Windows implementation

\end{fulllineitems}



\chapter{identifier module}
\label{\detokenize{identifier:module-identifier}}\label{\detokenize{identifier::doc}}\label{\detokenize{identifier:identifier-module}}\index{identifier (module)}\index{Identifier (class in identifier)}

\begin{fulllineitems}
\phantomsection\label{\detokenize{identifier:identifier.Identifier}}\pysigline{\sphinxstrong{class }\sphinxcode{identifier.}\sphinxbfcode{Identifier}}
Bases: \sphinxcode{object}

Article Identifier
Assigns a id kinda like sql auto increment works.
\index{assign() (identifier.Identifier method)}

\begin{fulllineitems}
\phantomsection\label{\detokenize{identifier:identifier.Identifier.assign}}\pysiglinewithargsret{\sphinxbfcode{assign}}{\emph{document}, \emph{article}}{}
Adds a \{``document'': document, ``article'': article\} to the ids list.
knows the position through the current id
:param document:
:param article:
:return:

\end{fulllineitems}

\index{retrieve() (identifier.Identifier method)}

\begin{fulllineitems}
\phantomsection\label{\detokenize{identifier:identifier.Identifier.retrieve}}\pysiglinewithargsret{\sphinxbfcode{retrieve}}{\emph{id}}{}
\end{fulllineitems}


\end{fulllineitems}



\chapter{ngrams module}
\label{\detokenize{ngrams:ngrams-module}}\label{\detokenize{ngrams::doc}}\label{\detokenize{ngrams:module-ngrams}}\index{ngrams (module)}\index{NGramIndex (class in ngrams)}

\begin{fulllineitems}
\phantomsection\label{\detokenize{ngrams:ngrams.NGramIndex}}\pysiglinewithargsret{\sphinxstrong{class }\sphinxcode{ngrams.}\sphinxbfcode{NGramIndex}}{\emph{length}}{}
Bases: \sphinxcode{object}
\index{index\_token() (ngrams.NGramIndex method)}

\begin{fulllineitems}
\phantomsection\label{\detokenize{ngrams:ngrams.NGramIndex.index_token}}\pysiglinewithargsret{\sphinxbfcode{index\_token}}{\emph{token}}{}
creates an index with the ngrams of each token.
:param token:
:return:

\end{fulllineitems}

\index{post\_filtering() (ngrams.NGramIndex method)}

\begin{fulllineitems}
\phantomsection\label{\detokenize{ngrams:ngrams.NGramIndex.post_filtering}}\pysiglinewithargsret{\sphinxbfcode{post\_filtering}}{\emph{wildcard\_input}, \emph{suggestions}}{}
checks if the suggestion actually matches the wildcard query
:param wildcard\_input:
:param suggestions:
:return:

\end{fulllineitems}

\index{suggestions() (ngrams.NGramIndex method)}

\begin{fulllineitems}
\phantomsection\label{\detokenize{ngrams:ngrams.NGramIndex.suggestions}}\pysiglinewithargsret{\sphinxbfcode{suggestions}}{\emph{ngrams}}{}
gets all the common tokens from an ngram list.
:param ngrams:
:return:

\end{fulllineitems}

\index{wildcard\_ngrams() (ngrams.NGramIndex method)}

\begin{fulllineitems}
\phantomsection\label{\detokenize{ngrams:ngrams.NGramIndex.wildcard_ngrams}}\pysiglinewithargsret{\sphinxbfcode{wildcard\_ngrams}}{\emph{wildcard\_input}}{}
creates the ngrams that based on the wildcard input.
:param wildcard\_input:
:return:

\end{fulllineitems}


\end{fulllineitems}

\index{get\_n\_grams() (in module ngrams)}

\begin{fulllineitems}
\phantomsection\label{\detokenize{ngrams:ngrams.get_n_grams}}\pysiglinewithargsret{\sphinxcode{ngrams.}\sphinxbfcode{get\_n\_grams}}{\emph{token}, \emph{grams\_count}}{}
returns the ngrams of a token
:param token: ex. results
:param grams\_count: ex. 2
:return: {[}'\$r', `re', `es', `su', `ul', `lt', `ts', `s\$'{]}

\end{fulllineitems}

\index{query\_combinations() (in module ngrams)}

\begin{fulllineitems}
\phantomsection\label{\detokenize{ngrams:ngrams.query_combinations}}\pysiglinewithargsret{\sphinxcode{ngrams.}\sphinxbfcode{query\_combinations}}{\emph{parts}}{}
creates a combination of all the possible queries
that could come out of wildcard query
:param parts:
:return:

\end{fulllineitems}



\chapter{terminal module}
\label{\detokenize{terminal:terminal-module}}\label{\detokenize{terminal::doc}}\label{\detokenize{terminal:module-terminal}}\index{terminal (module)}\index{clear\_console() (in module terminal)}

\begin{fulllineitems}
\phantomsection\label{\detokenize{terminal:terminal.clear_console}}\pysiglinewithargsret{\sphinxcode{terminal.}\sphinxbfcode{clear\_console}}{}{}
\end{fulllineitems}

\index{results\_menu() (in module terminal)}

\begin{fulllineitems}
\phantomsection\label{\detokenize{terminal:terminal.results_menu}}\pysiglinewithargsret{\sphinxcode{terminal.}\sphinxbfcode{results\_menu}}{\emph{results}}{}
\end{fulllineitems}

\index{show\_results() (in module terminal)}

\begin{fulllineitems}
\phantomsection\label{\detokenize{terminal:terminal.show_results}}\pysiglinewithargsret{\sphinxcode{terminal.}\sphinxbfcode{show\_results}}{\emph{results}, \emph{message=None}, \emph{menu=True}}{}
\end{fulllineitems}



\chapter{tokens module}
\label{\detokenize{tokens:module-tokens}}\label{\detokenize{tokens::doc}}\label{\detokenize{tokens:tokens-module}}\index{tokens (module)}\index{tokenizer() (in module tokens)}

\begin{fulllineitems}
\phantomsection\label{\detokenize{tokens:tokens.tokenizer}}\pysiglinewithargsret{\sphinxcode{tokens.}\sphinxbfcode{tokenizer}}{\emph{text}}{}
Tokenizes text into tokens.
:param text:
:return:

\end{fulllineitems}



\chapter{categorizer module}
\label{\detokenize{categorizer::doc}}\label{\detokenize{categorizer:categorizer-module}}\label{\detokenize{categorizer:module-categorizer}}\index{categorizer (module)}\index{FirstLetterSplitter (class in categorizer)}

\begin{fulllineitems}
\phantomsection\label{\detokenize{categorizer:categorizer.FirstLetterSplitter}}\pysiglinewithargsret{\sphinxstrong{class }\sphinxcode{categorizer.}\sphinxbfcode{FirstLetterSplitter}}{\emph{structure}, \emph{ngram\_index}}{}
Bases: \sphinxcode{object}

A tree per letter, digit, misc.
\index{find() (categorizer.FirstLetterSplitter method)}

\begin{fulllineitems}
\phantomsection\label{\detokenize{categorizer:categorizer.FirstLetterSplitter.find}}\pysiglinewithargsret{\sphinxbfcode{find}}{\emph{token}}{}
\end{fulllineitems}

\index{save() (categorizer.FirstLetterSplitter method)}

\begin{fulllineitems}
\phantomsection\label{\detokenize{categorizer:categorizer.FirstLetterSplitter.save}}\pysiglinewithargsret{\sphinxbfcode{save}}{}{}
\end{fulllineitems}

\index{size() (categorizer.FirstLetterSplitter method)}

\begin{fulllineitems}
\phantomsection\label{\detokenize{categorizer:categorizer.FirstLetterSplitter.size}}\pysiglinewithargsret{\sphinxbfcode{size}}{\emph{category=None}}{}
\end{fulllineitems}

\index{traverse() (categorizer.FirstLetterSplitter method)}

\begin{fulllineitems}
\phantomsection\label{\detokenize{categorizer:categorizer.FirstLetterSplitter.traverse}}\pysiglinewithargsret{\sphinxbfcode{traverse}}{}{}
\end{fulllineitems}

\index{update\_tree() (categorizer.FirstLetterSplitter method)}

\begin{fulllineitems}
\phantomsection\label{\detokenize{categorizer:categorizer.FirstLetterSplitter.update_tree}}\pysiglinewithargsret{\sphinxbfcode{update\_tree}}{\emph{document}}{}
\end{fulllineitems}

\index{visualize\_tree() (categorizer.FirstLetterSplitter method)}

\begin{fulllineitems}
\phantomsection\label{\detokenize{categorizer:categorizer.FirstLetterSplitter.visualize_tree}}\pysiglinewithargsret{\sphinxbfcode{visualize\_tree}}{}{}
\end{fulllineitems}


\end{fulllineitems}



\chapter{documents module}
\label{\detokenize{documents:module-documents}}\label{\detokenize{documents::doc}}\label{\detokenize{documents:documents-module}}\index{documents (module)}\index{document (class in documents)}

\begin{fulllineitems}
\phantomsection\label{\detokenize{documents:documents.document}}\pysiglinewithargsret{\sphinxstrong{class }\sphinxcode{documents.}\sphinxbfcode{document}}{\emph{doc\_name}, \emph{title}, \emph{text}}{}
Bases: \sphinxcode{object}
\index{get\_content() (documents.document method)}

\begin{fulllineitems}
\phantomsection\label{\detokenize{documents:documents.document.get_content}}\pysiglinewithargsret{\sphinxbfcode{get\_content}}{}{}
\end{fulllineitems}

\index{identifier() (documents.document method)}

\begin{fulllineitems}
\phantomsection\label{\detokenize{documents:documents.document.identifier}}\pysiglinewithargsret{\sphinxbfcode{identifier}}{}{}
\end{fulllineitems}


\end{fulllineitems}



\chapter{logic module}
\label{\detokenize{logic:logic-module}}\label{\detokenize{logic:module-logic}}\label{\detokenize{logic::doc}}\index{logic (module)}\index{exempt() (in module logic)}

\begin{fulllineitems}
\phantomsection\label{\detokenize{logic:logic.exempt}}\pysiglinewithargsret{\sphinxcode{logic.}\sphinxbfcode{exempt}}{\emph{a}, \emph{b}}{}
exempts b from a

\end{fulllineitems}

\index{intersect() (in module logic)}

\begin{fulllineitems}
\phantomsection\label{\detokenize{logic:logic.intersect}}\pysiglinewithargsret{\sphinxcode{logic.}\sphinxbfcode{intersect}}{\emph{a}, \emph{b}}{}
returns common items between two sets

\end{fulllineitems}

\index{union() (in module logic)}

\begin{fulllineitems}
\phantomsection\label{\detokenize{logic:logic.union}}\pysiglinewithargsret{\sphinxcode{logic.}\sphinxbfcode{union}}{\emph{a}, \emph{b}}{}
returns the sum of both lists without duplicates

\end{fulllineitems}



\chapter{porter module}
\label{\detokenize{porter:porter-module}}\label{\detokenize{porter::doc}}\label{\detokenize{porter:module-porter}}\index{porter (module)}
Porter Stemming Algorithm
This is the Porter stemming algorithm, ported to Python from the
version coded up in ANSI C by the author. It may be be regarded
as canonical, in that it follows the algorithm presented in

Porter, 1980, An algorithm for suffix stripping, Program, Vol. 14,
no. 3, pp 130-137,

only differing from it at the points maked --DEPARTURE-- below.

See also \url{http://www.tartarus.org/~martin/PorterStemmer}

The algorithm as described in the paper could be exactly replicated
by adjusting the points of DEPARTURE, but this is barely necessary,
because (a) the points of DEPARTURE are definitely improvements, and
(b) no encoding of the Porter stemmer I have seen is anything like
as exact as this version, even with the points of DEPARTURE!

Vivake Gupta (\href{mailto:v@nano.com}{v@nano.com})

Release 1: January 2001

Further adjustments by Santiago Bruno (\href{mailto:bananabruno@gmail.com}{bananabruno@gmail.com})
to allow word input not restricted to one word per line, leading
to:

release 2: July 2008
\index{PorterStemmer (class in porter)}

\begin{fulllineitems}
\phantomsection\label{\detokenize{porter:porter.PorterStemmer}}\pysigline{\sphinxstrong{class }\sphinxcode{porter.}\sphinxbfcode{PorterStemmer}}
Bases: \sphinxcode{object}
\index{cons() (porter.PorterStemmer method)}

\begin{fulllineitems}
\phantomsection\label{\detokenize{porter:porter.PorterStemmer.cons}}\pysiglinewithargsret{\sphinxbfcode{cons}}{\emph{i}}{}
cons(i) is TRUE \textless{}=\textgreater{} b{[}i{]} is a consonant.

\end{fulllineitems}

\index{cvc() (porter.PorterStemmer method)}

\begin{fulllineitems}
\phantomsection\label{\detokenize{porter:porter.PorterStemmer.cvc}}\pysiglinewithargsret{\sphinxbfcode{cvc}}{\emph{i}}{}
cvc(i) is TRUE \textless{}=\textgreater{} i-2,i-1,i has the form consonant - vowel - consonant
and also if the second c is not w,x or y. this is used when trying to
restore an e at the end of a short  e.g.
\begin{quote}

cav(e), lov(e), hop(e), crim(e), but
snow, box, tray.
\end{quote}

\end{fulllineitems}

\index{doublec() (porter.PorterStemmer method)}

\begin{fulllineitems}
\phantomsection\label{\detokenize{porter:porter.PorterStemmer.doublec}}\pysiglinewithargsret{\sphinxbfcode{doublec}}{\emph{j}}{}
doublec(j) is TRUE \textless{}=\textgreater{} j,(j-1) contain a double consonant.

\end{fulllineitems}

\index{ends() (porter.PorterStemmer method)}

\begin{fulllineitems}
\phantomsection\label{\detokenize{porter:porter.PorterStemmer.ends}}\pysiglinewithargsret{\sphinxbfcode{ends}}{\emph{s}}{}
ends(s) is TRUE \textless{}=\textgreater{} k0,...k ends with the string s.

\end{fulllineitems}

\index{m() (porter.PorterStemmer method)}

\begin{fulllineitems}
\phantomsection\label{\detokenize{porter:porter.PorterStemmer.m}}\pysiglinewithargsret{\sphinxbfcode{m}}{}{}
m() measures the number of consonant sequences between k0 and j.
if c is a consonant sequence and v a vowel sequence, and \textless{}..\textgreater{}
indicates arbitrary presence,
\begin{quote}

\textless{}c\textgreater{}\textless{}v\textgreater{}       gives 0
\textless{}c\textgreater{}vc\textless{}v\textgreater{}     gives 1
\textless{}c\textgreater{}vcvc\textless{}v\textgreater{}   gives 2
\textless{}c\textgreater{}vcvcvc\textless{}v\textgreater{} gives 3
....
\end{quote}

\end{fulllineitems}

\index{r() (porter.PorterStemmer method)}

\begin{fulllineitems}
\phantomsection\label{\detokenize{porter:porter.PorterStemmer.r}}\pysiglinewithargsret{\sphinxbfcode{r}}{\emph{s}}{}
r(s) is used further down.

\end{fulllineitems}

\index{setto() (porter.PorterStemmer method)}

\begin{fulllineitems}
\phantomsection\label{\detokenize{porter:porter.PorterStemmer.setto}}\pysiglinewithargsret{\sphinxbfcode{setto}}{\emph{s}}{}
setto(s) sets (j+1),...k to the characters in the string s, readjusting k.

\end{fulllineitems}

\index{stem() (porter.PorterStemmer method)}

\begin{fulllineitems}
\phantomsection\label{\detokenize{porter:porter.PorterStemmer.stem}}\pysiglinewithargsret{\sphinxbfcode{stem}}{\emph{p}, \emph{i}, \emph{j}}{}
In stem(p,i,j), p is a char pointer, and the string to be stemmed
is from p{[}i{]} to p{[}j{]} inclusive. Typically i is zero and j is the
offset to the last character of a string, (p{[}j+1{]} == `'). The
stemmer adjusts the characters p{[}i{]} ... p{[}j{]} and returns the new
end-point of the string, k. Stemming never increases word length, so
i \textless{}= k \textless{}= j. To turn the stemmer into a module, declare `stem' as
extern, and delete the remainder of this file.

\end{fulllineitems}

\index{step1ab() (porter.PorterStemmer method)}

\begin{fulllineitems}
\phantomsection\label{\detokenize{porter:porter.PorterStemmer.step1ab}}\pysiglinewithargsret{\sphinxbfcode{step1ab}}{}{}
step1ab() gets rid of plurals and -ed or -ing. e.g.

caresses  -\textgreater{}  caress
ponies    -\textgreater{}  poni
ties      -\textgreater{}  ti
caress    -\textgreater{}  caress
cats      -\textgreater{}  cat

feed      -\textgreater{}  feed
agreed    -\textgreater{}  agree
disabled  -\textgreater{}  disable

matting   -\textgreater{}  mat
mating    -\textgreater{}  mate
meeting   -\textgreater{}  meet
milling   -\textgreater{}  mill
messing   -\textgreater{}  mess

meetings  -\textgreater{}  meet

\end{fulllineitems}

\index{step1c() (porter.PorterStemmer method)}

\begin{fulllineitems}
\phantomsection\label{\detokenize{porter:porter.PorterStemmer.step1c}}\pysiglinewithargsret{\sphinxbfcode{step1c}}{}{}
step1c() turns terminal y to i when there is another vowel in the stem.

\end{fulllineitems}

\index{step2() (porter.PorterStemmer method)}

\begin{fulllineitems}
\phantomsection\label{\detokenize{porter:porter.PorterStemmer.step2}}\pysiglinewithargsret{\sphinxbfcode{step2}}{}{}
step2() maps double suffices to single ones.
so -ization ( = -ize plus -ation) maps to -ize etc. note that the
string before the suffix must give m() \textgreater{} 0.

\end{fulllineitems}

\index{step3() (porter.PorterStemmer method)}

\begin{fulllineitems}
\phantomsection\label{\detokenize{porter:porter.PorterStemmer.step3}}\pysiglinewithargsret{\sphinxbfcode{step3}}{}{}
step3() dels with -ic-, -full, -ness etc. similar strategy to step2.

\end{fulllineitems}

\index{step4() (porter.PorterStemmer method)}

\begin{fulllineitems}
\phantomsection\label{\detokenize{porter:porter.PorterStemmer.step4}}\pysiglinewithargsret{\sphinxbfcode{step4}}{}{}
step4() takes off -ant, -ence etc., in context \textless{}c\textgreater{}vcvc\textless{}v\textgreater{}.

\end{fulllineitems}

\index{step5() (porter.PorterStemmer method)}

\begin{fulllineitems}
\phantomsection\label{\detokenize{porter:porter.PorterStemmer.step5}}\pysiglinewithargsret{\sphinxbfcode{step5}}{}{}
step5() removes a final -e if m() \textgreater{} 1, and changes -ll to -l if
m() \textgreater{} 1.

\end{fulllineitems}

\index{vowelinstem() (porter.PorterStemmer method)}

\begin{fulllineitems}
\phantomsection\label{\detokenize{porter:porter.PorterStemmer.vowelinstem}}\pysiglinewithargsret{\sphinxbfcode{vowelinstem}}{}{}
vowelinstem() is TRUE \textless{}=\textgreater{} k0,...j contains a vowel

\end{fulllineitems}


\end{fulllineitems}



\chapter{querying module}
\label{\detokenize{querying::doc}}\label{\detokenize{querying:module-querying}}\label{\detokenize{querying:querying-module}}\index{querying (module)}\index{ExtendedPorterStemmer (class in querying)}

\begin{fulllineitems}
\phantomsection\label{\detokenize{querying:querying.ExtendedPorterStemmer}}\pysigline{\sphinxstrong{class }\sphinxcode{querying.}\sphinxbfcode{ExtendedPorterStemmer}}
Bases: \sphinxcode{core.queries.porter.PorterStemmer}

Extends the porter stemmer with a stem\_words method.
\index{stem\_words() (querying.ExtendedPorterStemmer method)}

\begin{fulllineitems}
\phantomsection\label{\detokenize{querying:querying.ExtendedPorterStemmer.stem_words}}\pysiglinewithargsret{\sphinxbfcode{stem\_words}}{\emph{words}}{}
stems each token in list of tokens
:param words
:return: list of stemmed tokens

\end{fulllineitems}


\end{fulllineitems}

\index{simple\_search() (in module querying)}

\begin{fulllineitems}
\phantomsection\label{\detokenize{querying:querying.simple_search}}\pysiglinewithargsret{\sphinxcode{querying.}\sphinxbfcode{simple\_search}}{\emph{stemmer}, \emph{token\_index}, \emph{queries}, \emph{multi\_query\_mode=False}}{}
Default search, finds results on each token, applies logic based on the
query then represents the results based on their tf-idf score.
:param stemmer:
:param token\_index: tokenindex
:param queries:
:param multi\_query\_mode:
:return:

\end{fulllineitems}

\index{sort\_query() (in module querying)}

\begin{fulllineitems}
\phantomsection\label{\detokenize{querying:querying.sort_query}}\pysiglinewithargsret{\sphinxcode{querying.}\sphinxbfcode{sort\_query}}{\emph{query}, \emph{default\_op=':and:'}}{}
sorts a query based on the boolean priorities.
:param: testing :or: not something :not: hype :or: random :not: python
:return: {[}'something', `:and:', `testing', `:or:', `not', `:or:', `random', `:not:', `hype', `:not:', `python'{]}

\end{fulllineitems}



\chapter{SGM module}
\label{\detokenize{SGM::doc}}\label{\detokenize{SGM:sgm-module}}\label{\detokenize{SGM:module-SGM}}\index{SGM (module)}\index{reuters\_SGM\_processor() (in module SGM)}

\begin{fulllineitems}
\phantomsection\label{\detokenize{SGM:SGM.reuters_SGM_processor}}\pysiglinewithargsret{\sphinxcode{SGM.}\sphinxbfcode{reuters\_SGM\_processor}}{\emph{file\_name}}{}
parses an sgm file articles into a list of documents.
:param file\_name:
:return:

\end{fulllineitems}



\chapter{tfidf module}
\label{\detokenize{tfidf:module-tfidf}}\label{\detokenize{tfidf::doc}}\label{\detokenize{tfidf:tfidf-module}}\index{tfidf (module)}\index{results\_tfidf (class in tfidf)}

\begin{fulllineitems}
\phantomsection\label{\detokenize{tfidf:tfidf.results_tfidf}}\pysigline{\sphinxstrong{class }\sphinxcode{tfidf.}\sphinxbfcode{results\_tfidf}}
Bases: \sphinxcode{object}

Keeps track of each document and it's associated ranking
\index{add\_documents() (tfidf.results\_tfidf method)}

\begin{fulllineitems}
\phantomsection\label{\detokenize{tfidf:tfidf.results_tfidf.add_documents}}\pysiglinewithargsret{\sphinxbfcode{add\_documents}}{\emph{doc\_object}}{}
Adds documents to the docs dictionary
:param doc\_object:
:return:

\end{fulllineitems}

\index{add\_idf() (tfidf.results\_tfidf method)}

\begin{fulllineitems}
\phantomsection\label{\detokenize{tfidf:tfidf.results_tfidf.add_idf}}\pysiglinewithargsret{\sphinxbfcode{add\_idf}}{\emph{idf}, \emph{term}}{}
Adds a term's idf to the idf dictionary
:param idf:
:param term:
:return:

\end{fulllineitems}

\index{calc\_tf\_idf() (tfidf.results\_tfidf method)}

\begin{fulllineitems}
\phantomsection\label{\detokenize{tfidf:tfidf.results_tfidf.calc_tf_idf}}\pysiglinewithargsret{\sphinxbfcode{calc\_tf\_idf}}{\emph{documents}}{}
Calculates each documents tf-idf ranking and then returns the
tf-idf dictionary
:param documents:
:return:

\end{fulllineitems}


\end{fulllineitems}



\chapter{Indices and tables}
\label{\detokenize{index:indices-and-tables}}\begin{itemize}
\item {} 
\DUrole{xref,std,std-ref}{genindex}

\item {} 
\DUrole{xref,std,std-ref}{modindex}

\item {} 
\DUrole{xref,std,std-ref}{search}

\end{itemize}


\renewcommand{\indexname}{Python Module Index}
\begin{sphinxtheindex}
\def\bigletter#1{{\Large\sffamily#1}\nopagebreak\vspace{1mm}}
\bigletter{b}
\item {\sphinxstyleindexentry{btree}}\sphinxstyleindexpageref{btree:\detokenize{module-btree}}
\indexspace
\bigletter{c}
\item {\sphinxstyleindexentry{categorizer}}\sphinxstyleindexpageref{categorizer:\detokenize{module-categorizer}}
\indexspace
\bigletter{d}
\item {\sphinxstyleindexentry{documents}}\sphinxstyleindexpageref{documents:\detokenize{module-documents}}
\indexspace
\bigletter{g}
\item {\sphinxstyleindexentry{getch}}\sphinxstyleindexpageref{getch:\detokenize{module-getch}}
\indexspace
\bigletter{i}
\item {\sphinxstyleindexentry{identifier}}\sphinxstyleindexpageref{identifier:\detokenize{module-identifier}}
\indexspace
\bigletter{l}
\item {\sphinxstyleindexentry{logic}}\sphinxstyleindexpageref{logic:\detokenize{module-logic}}
\indexspace
\bigletter{n}
\item {\sphinxstyleindexentry{ngrams}}\sphinxstyleindexpageref{ngrams:\detokenize{module-ngrams}}
\indexspace
\bigletter{p}
\item {\sphinxstyleindexentry{porter}}\sphinxstyleindexpageref{porter:\detokenize{module-porter}}
\indexspace
\bigletter{q}
\item {\sphinxstyleindexentry{querying}}\sphinxstyleindexpageref{querying:\detokenize{module-querying}}
\indexspace
\bigletter{s}
\item {\sphinxstyleindexentry{SGM}}\sphinxstyleindexpageref{SGM:\detokenize{module-SGM}}
\indexspace
\bigletter{t}
\item {\sphinxstyleindexentry{terminal}}\sphinxstyleindexpageref{terminal:\detokenize{module-terminal}}
\item {\sphinxstyleindexentry{tfidf}}\sphinxstyleindexpageref{tfidf:\detokenize{module-tfidf}}
\item {\sphinxstyleindexentry{tokens}}\sphinxstyleindexpageref{tokens:\detokenize{module-tokens}}
\end{sphinxtheindex}

\renewcommand{\indexname}{Index}
\printindex
\end{document}